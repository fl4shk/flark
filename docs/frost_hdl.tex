\documentclass{article}

\usepackage{graphicx}
\usepackage{float}
\usepackage{fancyvrb}
\usepackage[T1]{fontenc}
\usepackage{lmodern}



\title{Frost HDL}
\date{2018-10-19}
\author{Andrew Clark (FL4SHK)}

%% Hide section, subsection, and subsubsection numbering
%\renewcommand{\thesection}{}
%\renewcommand{\thesubsection}{}
%\renewcommand{\thesubsubsection}{}


%Figures
%\begin{figure}[H]
%	\includegraphics[width=\linewidth]{example.png}
%\end{figure}

% Verbatim text
%\VerbatimInput{main.cpp}

%% Fixed-width text
%\texttt{module FullAdder(input logic a, b, c_in, output logic s, c_out);}
%% Bold
%\textbf{green eggs}
%% Italic
%\textit{and}
%% Underline
%\underline{eggs}

%% Non-numbered list
%\begin{itemize}
%\item item 0
%\item item 1
%\end{itemize}

%% Numbered list
%\begin{enumerate}
%\item item 0
%\item item 1
%\end{enumerate}

%% Spaces and new lines
%LaTeX ignores extra spaces and new lines.
%Place \\ at the end of a line to create a new line (but not create a new
%paragraph)

%% Use "\noindent" to prevent a paragraph from indenting

%% Tables
%\begin{table}[H]
%	\begin{center}
%		\caption{Results for \texttt{blocksPerGrid = 5}}:
%		\label{tab:table0}
%		\begin{tabular}{c|c}
%			\textbf{\texttt{threadsPerBlock}}
%				& \textbf{\texttt{scaling()} Running Time (us)}\\
%			\hline
%			32 & 156.39\\
%			64 & 163.59\\
%			128 & 155.62\\
%			256 & 155.56\\
%			512 & 161.57\\
%			1024 & 166.85\\
%		\end{tabular}
%	\end{center}
%\end{table}


\begin{document}
	\maketitle
	\pagenumbering{gobble}
	\newpage
	\pagenumbering{arabic}

	\section{Introduction}
		\begin{itemize}
		\item This is a "high level" hardware description language.  The
		main goal is to take features from synthesizeable SystemVerilog
		(which aren't in Verilog) and expand upon them, and also to have
		direct support for yosys's formal verification abilities.
		\item The language syntax is also set up differently from both
		Verilog and SystemVerilog to eliminate some of my pet peeves.
		\item This language compiles to Verilog-2001, but support for
		compiling to SystemVerilog may be added later.  On that note, this
		language is ONLY compiled, and does not have its own simulator.
		This allows me to piggy back off of Verilog's semantics.
		\end{itemize}
	\section{Language Details/Features}
		\begin{itemize}
		\item Syntax changes
			\begin{itemize}
			\item The biggest syntax change is that \texttt{\{} and
			\texttt{\}} are used to indicate code blocks, \textbf{not}
			\texttt{begin} and \texttt{end}.  I will also have \texttt{\{}
			and \texttt{\}} be used to indicate \texttt{module} contents,
			so \texttt{endmodule} is gone.
			\item As such, concatenation needs new syntax.  I have simply
			chosen \texttt{\$concat()} for the new syntax.
			\item Replication also has a new syntax:
			\texttt{\$replicate()}.
			\end{itemize}
		\item What doesn't exist
			\begin{itemize}
			\item Verilog's \texttt{fork} and \texttt{join} constructs are
			not supported at this time, but may be added later, especially
			if I can figure out what they're actually used for.  They do
			not appear to be synthesizeable constructs anyway.
			\end{itemize}
		\item \texttt{package}s
			\begin{itemize}
			\item Nested \texttt{package}s are supported.  This may be
			something that SystemVerilog proper supports, but Icarus
			Verilog's implementation of SystemVerilog's \texttt{package}s
			sure does not support nested \texttt{package}s.
			\end{itemize}
		\item \texttt{struct}s
			\begin{itemize}
			\item First and foremost, unlike in SystemVerilog,
			\texttt{struct}s can be \texttt{parameter}ized.
			\item A \texttt{struct} can have member \texttt{task}s and
			member \texttt{function}s.  These operate basically how you
			would expect.
			\item \texttt{public} and \texttt{private} are supported as
			well.
			\item Constructors and destructors are not supported.
			Destructors in particular do not make sense to support, but
			constructors \textbf{may} be added for \texttt{struct}s that
			end up implemented as \texttt{reg} vectors in the generated
			code.
			\item For code generation, \texttt{struct}s are simply compiled
			to either \texttt{wire} or \texttt{reg} vectors, with a
			\texttt{struct}'s data members simply being different slices of
			the generated \texttt{wire} or \texttt{reg} vector.
			\item Note that \texttt{struct}s in this language are very
			similar to SystemVerilog's \texttt{packed struct}s, and
			SystemVerilog's non-\texttt{packed struct}s simply do not
			exist.
			\end{itemize}
		\item \texttt{union}s
			\begin{itemize}
			\item \texttt{union}s may or may not be added.  I honestly
			\textbf{might} add them, but it's not as high a priority as
			other things.
			\end{itemize}
		\item \texttt{interface}s
			\begin{itemize}
			\item These are intended to mimic SystemVerilog's
			\texttt{interface}s, though I may make some changes.
			\item Having not had much access to SystemVerilog's
			\texttt{interface}s myself, I may have to review how they work
			first before I can implement them.
			\end{itemize}
		\item \texttt{macro}s
			\begin{itemize}
			\item I really want to support a useful macro system, perhaps
			in the form of a preprocessor.  This is perhaps my replacement
			for Verilog's \texttt{generate}, but I might need to support
			\texttt{generate} as well.
			\end{itemize}
		\item Formal Verification
			\begin{itemize}
			\item yosys's formal verification constructs are directly
			supported and will simply be inserted directly into the
			generated code.
			\item The constructs \texttt{assert()}, \texttt{assume()},
			\texttt{assert property}, \texttt{assume property}, etc. are
			all supported directly.
			\end{itemize}
		\end{itemize}

\end{document}
