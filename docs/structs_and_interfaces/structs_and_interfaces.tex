\documentclass{article}

\usepackage{graphicx}
\usepackage{float}
\usepackage{fancyvrb}
\usepackage[T1]{fontenc}
\usepackage{lmodern}
\usepackage{setspace}
\usepackage[nottoc]{tocbibind}
\usepackage[font=large]{caption}
\usepackage{framed}
\usepackage{tabularx}
\usepackage{amsmath}
\usepackage{hyperref}
\usepackage[backend=biber,sorting=none]{biblatex}
%%\usepackage[
%%	backend=biber,
%%	style=ieee,
%%	sorting=none
%%]{biblatex}
%\addbibresource{project_refs.bib}

\title{Frost HDL \texttt{struct}s and \texttt{interface}s}
%\date{2018-11-08}
\author{Andrew Clark}

%% Hide section, subsection, and subsubsection numbering
%\renewcommand{\thesection}{}
%\renewcommand{\thesubsection}{}
%\renewcommand{\thesubsubsection}{}

% Alternative form of doing section stuff
\renewcommand{\thesection}{}
\renewcommand{\thesubsection}{\arabic{section}.\arabic{subsection}}
\makeatletter
\def\@seccntformat#1{\csname #1ignore\expandafter\endcsname\csname the#1\endcsname\quad}
\let\sectionignore\@gobbletwo
\let\latex@numberline\numberline
\def\numberline#1{\if\relax#1\relax\else\latex@numberline{#1}\fi}
\makeatother

\makeatletter
\renewcommand\tableofcontents{%
    \@starttoc{toc}%
}
\makeatother

%Figures
%\begin{figure}[H]
%	\includegraphics[width=\linewidth]{example.png}
%	\caption{Example Figure Title}
%	\label{fig:example}
%\end{figure}

% Verbatim text
%\VerbatimInput{main.cpp}

%% Fixed-width text
%\texttt{module FullAdder(input logic a, b, c_in, output logic s, c_out);}
%% Bold
%\textbf{green eggs}
%% Italic
%\textit{and}
%% Underline
%\underline{eggs}

%% Non-numbered list
%\begin{itemize}
%\item item 0
%\item item 1
%\end{itemize}

%% Numbered list
%\begin{enumerate}
%\item item 0
%\item item 1
%\end{enumerate}

%% Spaces and new lines
%LaTeX ignores extra spaces and new lines.
%Place \\ at the end of a line to create a new line (but not create a new
%paragraph)

%% Use "\noindent" to prevent a paragraph from indenting

%% Tables
%\begin{table}[H]
%	\begin{center}
%		\caption{Results for \texttt{blocksPerGrid = 5}}:
%		\label{tab:table0}
%		\begin{tabular}{c|c}
%			\textbf{\texttt{threadsPerBlock}}
%				& \textbf{\texttt{scaling()} Running Time (us)}\\
%			\hline
%			32 & 156.39\\
%			64 & 163.59\\
%			128 & 155.62\\
%			256 & 155.56\\
%			512 & 161.57\\
%			1024 & 166.85\\
%		\end{tabular}
%	\end{center}
%\end{table}

\begin{document}
	\maketitle
	\pagenumbering{gobble}
	\newpage
	\pagenumbering{arabic}


	\doublespacing
	%\section{Abstract}
	%\setcounter{section}{-1}

	\section{Overview}
	\texttt{struct}s and \texttt{interface}s are two major features being
	borrowed from SystemVerilog.  \texttt{struct}s in Frost HDL will
	have more features than those in SystemVerilog.  However,
	\texttt{interface}s in Frost HDL will have fewer features than
	\texttt{interface}s in SystemVerilog.


	\section{\texttt{struct}s}
	Notably, \texttt{struct}s in Frost HDL can be \texttt{parameter}ized.
	\texttt{struct}s in SystemVerilog cannot be \texttt{parameter}ized.


	\subsection{The Three Forms}
	\texttt{struct}s come in three forms in Frost HDL:  \texttt{packed},
	\texttt{unpacked}, and \texttt{splitvar}.

	\texttt{packed} and \texttt{unpacked} \texttt{struct}s are similar to
	the \texttt{struct}s of SystemVerilog, but \texttt{unpacked} is a
	required keyword if that type of \texttt{struct} is desired in Frost
	HDL.

	\texttt{splitvar} \texttt{struct}s in Frost HDL are like an inbetween
	of \texttt{packed} and \texttt{unpacked} \texttt{struct}s.

	Additionally, \texttt{struct}s in general can be used \textit{as}
	\texttt{parameter}s of other entities.

	\subsubsection{\texttt{packed} \texttt{struct}s}
	\texttt{packed} \texttt{struct}s are the most restrictive type of
	\texttt{struct} in Frost HDL in terms of what their elements can be.

	They can placed on \texttt{module} ports and can be placed inside of
	\texttt{interface}s.
	
	They cannot store arrays of any type, but they can store scalars of
	built-in types and scalars of other \texttt{packed} \texttt{struct}s.

	Also, they compile into plain old Verilog vectors (of type
	\texttt{wire} or \texttt{reg}), with different (non-overlapping) slices
	of the vector indicating the element of the \texttt{packed}
	\texttt{struct}.
	
	They can be used as vectors within Frost HDL source code, as well, but
	this will require manual casting.

	\subsubsection{\texttt{unpacked} \texttt{struct}s}
	\texttt{unpacked} \texttt{struct}s can not be placed on \texttt{module}
	ports or inside of \texttt{interface}s.

	However, other than this restriction, they are the most flexible type
	of \texttt{struct} in Frost HDL.  They can contain arrays, even arrays
	of other other \texttt{struct}s.

	An array of \texttt{unpacked} \texttt{struct}s will be compiled into a
	\texttt{struct}ure of arrays.  This is the primary method by which one
	can use multi-dimensional arrays in Frost HDL.  Simply use an array of
	\texttt{unpacked} \texttt{struct}s where each \texttt{struct} contains
	another array.  This will produce a multi-dimensional array in the
	generated Verilog.  This is admittedly a little inconvenient, so
	direct support for multi-dimensional arrays \textit{may} be added to
	the language later.

	In some simple cases cases, an \texttt{unpacked} \texttt{struct}
	\textit{may} be compiled into a vector in the generated Verilog.

	\subsubsection{\texttt{splitvar} \texttt{struct}s}
	\texttt{splitvar} \texttt{struct}s are inbetween \texttt{packed} and
	\texttt{unpacked} \texttt{struct}s in terms of functionality.

	They can be placed on ports.

	They cannot be treated as vectors, but they can contain arrays of
	built-in types and arrays of \texttt{packed} \texttt{struct}s.  They
	can contain non-array elements of these as well.

	The elements of a \texttt{splitvar} \texttt{struct} are compiled into
	separate variables in the generated Verilog, including if placed on
	\texttt{module} ports or inside of an \texttt{interface}.  These
	variables will share the same port direction if their \texttt{splitvar}
	\texttt{struct} is used as \texttt{module} ports.


	\section{\texttt{interface}s}
	These are similar to basic \texttt{interface}s in SystemVerilog.
	Unlike in SystemVerilog, however, they can not have their own ports.

	\texttt{modport} must be used if an \texttt{interface} is used for a
	\texttt{module} port.  A \texttt{modport} will be "flattened" into
	Verilog \texttt{module} ports.

	In the \texttt{module} that an \texttt{interface} is instantiated, the
	component variables of that \texttt{interface} will be implemented, in
	the generated Verilog, as separate variables, similar to what
	\textit{may} happen with \texttt{unpacked} \texttt{struct}s and what
	\textit{always} happens with \texttt{splitvar} \texttt{struct}s.

	\texttt{interface}s cannot be instantiated inside of other
	\texttt{interface}s, and \texttt{unpacked} \texttt{struct}s may not be
	instantiated inside of \texttt{interface}s.

	All other types (built-in; \texttt{enum}s; \texttt{packed}
	\texttt{struct}s; \texttt{splitvar} \texttt{struct}s) are fair game,
	and they may be instantiated inside of \texttt{interface}s.

	\texttt{interface}s may be \texttt{parameter}ized, and they may contain
	\texttt{function}s and \texttt{task}s.

	Note that any Frost HDL language constructs that are not mentioned here
	are probably \textit{not} going to be allowed inside of an
	\texttt{interface}.

	What I'm not sure about is if I want to support polymorphism via
	\texttt{interface}s and the name of a \texttt{modport}.  This seems
	like a desirable feature, and it would be feasible to implement because
	the compiler can see which \texttt{interface}s are used where.

	Perhaps some form of implementation inheritance for \texttt{interface}s
	could be supported as well.

	%\printbibliography[heading=bibnumbered,title={Bibliography}]

\end{document}

