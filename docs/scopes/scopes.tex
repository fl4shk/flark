\documentclass{article}

\usepackage{graphicx}
\usepackage{float}
\usepackage{fancyvrb}
\usepackage[T1]{fontenc}
\usepackage{lmodern}
\usepackage{setspace}
\usepackage[nottoc]{tocbibind}
\usepackage[font=large]{caption}
\usepackage{framed}
\usepackage{tabularx}
\usepackage{amsmath}
\usepackage{hyperref}
\usepackage[backend=biber,sorting=none]{biblatex}
%%\usepackage[
%%	backend=biber,
%%	style=ieee,
%%	sorting=none
%%]{biblatex}
%\addbibresource{project_refs.bib}

\title{Frost HDL Compiler Scopes Implementation}
%\date{2018-11-08}
\author{Andrew Clark}

%% Hide section, subsection, and subsubsection numbering
%\renewcommand{\thesection}{}
%\renewcommand{\thesubsection}{}
%\renewcommand{\thesubsubsection}{}

% Alternative form of doing section stuff
\renewcommand{\thesection}{}
\renewcommand{\thesubsection}{\arabic{section}.\arabic{subsection}}
\makeatletter
\def\@seccntformat#1{\csname #1ignore\expandafter\endcsname\csname the#1\endcsname\quad}
\let\sectionignore\@gobbletwo
\let\latex@numberline\numberline
\def\numberline#1{\if\relax#1\relax\else\latex@numberline{#1}\fi}
\makeatother

\makeatletter
\renewcommand\tableofcontents{%
    \@starttoc{toc}%
}
\makeatother

%Figures
%\begin{figure}[H]
%	\includegraphics[width=\linewidth]{example.png}
%	\caption{Example Figure Title}
%	\label{fig:example}
%\end{figure}

% Verbatim text
%\VerbatimInput{main.cpp}

%% Fixed-width text
%\texttt{module FullAdder(input logic a, b, c_in, output logic s, c_out);}
%% Bold
%\textbf{green eggs}
%% Italic
%\textit{and}
%% Underline
%\underline{eggs}

%% Non-numbered list
%\begin{itemize}
%\item item 0
%\item item 1
%\end{itemize}

%% Numbered list
%\begin{enumerate}
%\item item 0
%\item item 1
%\end{enumerate}

%% Spaces and new lines
%LaTeX ignores extra spaces and new lines.
%Place \\ at the end of a line to create a new line (but not create a new
%paragraph)

%% Use "\noindent" to prevent a paragraph from indenting

%% Tables
%\begin{table}[H]
%	\begin{center}
%		\caption{Results for \texttt{blocksPerGrid = 5}}:
%		\label{tab:table0}
%		\begin{tabular}{c|c}
%			\textbf{\texttt{threadsPerBlock}}
%				& \textbf{\texttt{scaling()} Running Time (us)}\\
%			\hline
%			32 & 156.39\\
%			64 & 163.59\\
%			128 & 155.62\\
%			256 & 155.56\\
%			512 & 161.57\\
%			1024 & 166.85\\
%		\end{tabular}
%	\end{center}
%\end{table}

\begin{document}
	\maketitle
	\pagenumbering{gobble}
	\newpage
	\pagenumbering{arabic}


	\doublespacing
	%\section{Abstract}
	%\setcounter{section}{-1}

	\singlespacing

	\section{Table of Contents}
	\tableofcontents

	\newpage

	\section{Scope Types and Containable Ones, Too}
	\begin{itemize}
		\item \texttt{module}
		\begin{itemize}
			\item \texttt{struct} / \texttt{class}
			%\item \texttt{union}
			\item \texttt{enum}
			\item \texttt{function} / \texttt{task}
			\item Behavioral code blocks
			\begin{itemize}
				\item \texttt{initial}
				\item \texttt{always}
				\item \texttt{always\_comb}
				\item \texttt{always\_seq}
				\item \texttt{generate}
			\end{itemize}
			\item \texttt{module} instantiations
		\end{itemize}

		\item \texttt{package}
		\begin{itemize}
			\item (Other) \texttt{package}
			\item \texttt{struct} / \texttt{class}
			%\item \texttt{union}
			\item \texttt{enum}
			\item \texttt{function} / \texttt{task}
		\end{itemize}

		\item \texttt{struct} / \texttt{class}
		\begin{itemize}
			\item (Other) \texttt{struct} / \texttt{class}
			%\item \texttt{union}
			\item \texttt{enum}
			\item \texttt{function} / \texttt{task}
		\end{itemize}

		%\item \texttt{union}
		\item \texttt{enum}

		\item \texttt{function} / \texttt{task}

		\item Behavioral code blocks
		\begin{itemize}
			\item \texttt{initial}
			\item \texttt{always}
			\item \texttt{always\_comb}
			\item \texttt{always\_seq}
			\item \texttt{generate}
		\end{itemize}
	\end{itemize}

	\newpage
	\doublespacing

	\section{Overall Handling of Scopes}
	First, it is worthwhile to mention that all source files provided to
	the compiler will be treated as if they had been concatenated together
	into one big string.  This will work in part because the compiler will
	use multiple passes to find all scopes.  As such, there is no per-file
	scoping; the global scope in one source file is part of the same global
	scope as any other source file.

	Each scope is of a certain type, such as \texttt{module},
	\texttt{struct}, or \texttt{package}.  Each type scope has different
	contexts in which it can appear, and there is some support for nested
	scoping.

	\section{Types of Scopes}

	\subsection{\texttt{module}s}

	This is the primary type of scope, without which no Verilog code will
	be generated.  The other types of scopes are intended to be used as
	helpers for this type of scope.  These are analagous to Verilog
	\texttt{module}s.

	\texttt{parameter}ized \texttt{module}s are extremely useful, and will
	be supported directly.  However, they may generate Verilog code for
	non-\texttt{parameter}ized \texttt{module}s so that Frost HDL can use
	its own semantics for them.

	A \texttt{module} can only be placed at global scope, at least in the
	initial version of the language.  Because \texttt{parameter}ized
	\texttt{module}s are probably going to have their names mangled, it
	will probably be easy to allow \texttt{module} definitions in scopes
	other than global scope.  If \texttt{module} definitions are ever
	allowed in non-global scopes, such scopes will probably have to be
	either a \texttt{package} or another \texttt{module}.

	\subsection{\texttt{enum}s}

	These are very similar to C++'s \texttt{enum class} construct, perhaps
	without strongly.


	%\printbibliography[heading=bibnumbered,title={Bibliography}]

\end{document}

